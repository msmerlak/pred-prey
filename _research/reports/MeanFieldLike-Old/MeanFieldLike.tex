\documentclass[10pt]{article}

\usepackage[utf8x]{inputenc}
\usepackage[english,italian]{babel}
\usepackage{graphicx} 
\usepackage{booktabs}
\usepackage{caption}
\usepackage{textgreek}
\usepackage{tabularx}
\usepackage{amsmath,amssymb,stackengine}
\usepackage{authblk}
\usepackage{xcolor}


\usepackage{geometry} % to change the page dimensions
\geometry{a4paper} % or letterpaper (US) or a5paper or....
% \geometry{margin=2in} % for example, change the margins to 2 inches all round
% \geometry{landscape} % set up the page for landscape
%   read geometry.pdf for detailed page layout information

\usepackage{graphicx} % support the \includegraphics command and options

% \usepackage[parfill]{parskip} % Activate to begin paragraphs with an empty line rather than an indent

%%% PACKAGES
\usepackage{booktabs} % for much better looking tables
\usepackage{array} % for better arrays (eg matrices) in maths
\usepackage{paralist} % very flexible & customisable lists (eg. enumerate/itemize, etc.)
\usepackage{verbatim} % adds environment for commenting out blocks of text & for better verbatim
\usepackage{subfig} % make it possible to include more than one captioned figure/table in a single float
% These packages are all incorporated in the memoir class to one degree or another...

%%% HEADERS & FOOTERS
\usepackage{fancyhdr} % This should be set AFTER setting up the page geometry
\pagestyle{fancy} % options: empty , plain , fancy
\renewcommand{\headrulewidth}{0pt} % customise the layout...
\lhead{}\chead{}\rhead{}
\lfoot{}\cfoot{\thepage}\rfoot{}

\usepackage{sectsty}
\allsectionsfont{\sffamily\mdseries\upshape} 

\usepackage[nottoc,notlof,notlot]{tocbibind}
\usepackage[titles,subfigure]{tocloft} 

\renewcommand{\cftsecfont}{\rmfamily\mdseries\upshape}
\renewcommand{\cftsecpagefont}{\rmfamily\mdseries\upshape} 
\newcommand*{\hham}{\mathcal{H}}
\newcommand*{\xx}{\vec{x}}
\newcommand*{\kk}{\vec{k}}
\newcommand*{\qq}{\vec{q}}
\newcommand*{\p}{\varphi}
\newcommand*{\zpart}{\mathcal{Z}}
\newcommand*{\w}{\Bigl}
\newcommand*{\lapint}{\int_{a-i\infty}^{a+i\infty}\frac{dz}{2\pi i}}

\definecolor{carmine}{rgb}{0.59, 0.0, 0.09}


\selectlanguage{english}

\title{{\bf Phase transition in a simple food web}}

\author{Onofrio Mazzarisi}

\affil{{\it Max Planck Institute for Mathematics in the Sciences, Leipzig, Germany}}

\begin{document}

\selectlanguage{english}

\maketitle

\begin{abstract}
This is a simple mean field (in the sense that lack spatial structure) model for a food web of one prey species
and one predator species. I designed it having in mind the principal properties of the one in the draft
but in a simplified version, in particular the evolution part is absent. 
The model shows a phase transition in the predator (probability) density \(\pi\) at a critical
value \(\lambda_c\) of the resources generation rate \(\lambda\).
\end{abstract} 

\section{The model}

To build the model I refer to the properties of the one of the draft, in particular I think the resources
distribution among preys and predators and their role in the reproductive and death rate is a relevant
feature to observe the emergence of the transition with respect to the parameter \(\lambda\) which
controls the growth rate of the resources. Therefore I keep track of the distribution of
resources present 'inside' the preys \(r_p\) and 'inside' the predators \(r_\pi\) as
independent variable with respect to the density of preys \(p\) and predators \(\pi\).

The dynamical variables of the problem are
\begin{itemize}
\item
{
	\(r_f\), the density of free resources
}
\item
{
	\(r_p\), the density of resources collected by preys
} 
\item
{
	\(r_{\pi}\), the density of the resources collected by predators
}
\item
{
	\(p\), the density of preys
}
\item
{
	\(\pi\), the density predators
}
\end{itemize}
We have a free parameter \(\lambda\), which is the constant {\it growth} rate at which new free resources are generated,
and fixed parameters which are the {\it voracity} \(v_p\) and \(v_{\pi}\), which regulates how good
prey and predators are able to capture respectively free resources and preys and the
{\it metabolisms} \(m_p\) and \(m_{\pi}\) which controls both the reproduction and death rates of
prey and predators.
The master equation reads
\begin{align}
\dot{r}_f&=\lambda -v_pr_fp \label{eq:r_f}\\
\dot{r}_p&= r_fv_pp-m_pr_p-v_{\pi}r_p\pi \label{eq:r_p}\\
\dot{r}_{\pi}&= v_{\pi}r_p\pi-m_{\pi}r_{\pi} \label{eq:r_pi}\\
\dot{p}&= r_pm_p-m_pp-v_{\pi}\pi p \label{eq:p}\\
\dot{\pi}&= r_{\pi}m_{\pi}-m_{\pi}\pi \label{eq:pi}.
\end{align}
The free resources \(r_f\) increase with constant rate \(\lambda\) and get consumed
with rate \(v_pr_fp\) which accounts for how good are the preys to collect the free resources and how
abundant are both, we obtain Eq.~(\ref{eq:r_f}). The resources of the preys \(r_p\)
increase consistently with the decrease of the free resources and they are depleted
by the metabolism of the preys \(m_p(r_p/p)p=m_pr_p\) and by the amount that the predators
take from the preys \(v_{\pi}\pi(r_p/p)p=v_{\pi}\pi r_p\) and one gets Eq.~(\ref{eq:r_p}).
Similar reasoning brings to Eq.~(\ref{eq:r_pi}). The density of preys is increased or decreased
by their metabolism dependently on the abundance of resources they collected, they reproduce if there is more
then one resource per individual and they die if there is less, the term 
\((r_p/p-1)m_pp=m_pr_p-m_pp\) accounts for this. Moreover the density of preys is reduced by predation
which is accounted for with the term \(v_{\pi}\pi\), Eq.~(\ref{eq:p}) follows. Equation~(\ref{eq:pi})
is derived with the same arguments.


\section{Stationary solution}

If we look for a stationary solution we obtain from Eq.~(\ref{eq:r_f}) \(r_f^s=\lambda/v_pp\),
from Eq.~(\ref{eq:pi}) \(r_{\pi}^s=\pi\) and
then from Eq.~(\ref{eq:r_pi}) \(r_p^s=m_{\pi}/v_{\pi}\).
We are left with
\begin{align}
0&=\frac{m_{\pi}m_p}{v_{\pi}}-m_pp-v_{\pi}\pi p \\
0&=\lambda-\frac{m_pm_{\pi}}{v_{\pi}}-m_{\pi}\pi.
\end{align}
From the first one we obtain
\begin{equation}
\pi=\frac{m_{\pi}m_p}{v^2_{\pi}p}-\frac{m_p}{v_{\pi}},
\end{equation}
and from the second one
\begin{equation}
p^s=\frac{m^2_{\pi}m_p}{\lambda v^2_{\pi}}.
\label{eq:solution-preys}
\end{equation}
Therefore, for given value of the fixed parameters, we have a stable finite solution for the density
of the preys for every \(\lambda\). For the predators this is not true and we can see that,
substituting now Eq.~(\ref{eq:solution-preys}) in the solution for \(\pi\) we obtain
\begin{equation}
\pi^s=\frac{1}{m_{\pi}}\Bigl(\lambda-\frac{m_pm_{\pi}}{v_{\pi}}\Bigl).
\end{equation}

A meaningful finite stationary solution for the predator density exist only if 
\(\lambda>\lambda_c\), with the critical value given by
\begin{equation}
\lambda_c=\frac{m_pm_{\pi}}{v_{\pi}}.
\end{equation}

\bibliography{Bib/library.bib}

\bibliographystyle{unsrt}

\end{document}